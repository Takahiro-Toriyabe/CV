%%%%%%%%%%%%%%%%%%%%%%%%%%%%%%%%%%%%%%%%%
% Compact Academic CV
% LaTeX Template
% Version 1.0 (10/6/2012)
%
% This template has been downloaded from:
% http://www.LaTeXTemplates.com
%
% Original author:
% Dario Taraborelli (http://nitens.org/taraborelli/home)
%
% License:
% CC BY-NC-SA 3.0 (http://creativecommons.org/licenses/by-nc-sa/3.0/)
%
% Important:
% This template needs to be compiled using XeLaTeX
%
% Note: this template has the option to use the Hoefler Text font, see the
% font configurations section below for instructions on using this font
%
%%%%%%%%%%%%%%%%%%%%%%%%%%%%%%%%%%%%%%%%%

%----------------------------------------------------------------------------------------
%	PACKAGES AND OTHER DOCUMENT CONFIGURATIONS
%----------------------------------------------------------------------------------------

\documentclass[11pt, a4paper]{article} % Document font size and paper size

\usepackage{fontspec} % Allows the use of OpenType fonts

\usepackage{geometry} % Allows the configuration of document margins
\geometry{a4paper, textwidth=5.5in, textheight=8.5in, marginparsep=7pt, marginparwidth=.6in} % Document margin settings
\setlength\parindent{0in} % Remove paragraph indentation

\usepackage[usenames,dvipsnames]{xcolor} % Custom colors

\usepackage{sectsty} % Allows changing the font options for sections in a document
\usepackage[normalem]{ulem} % Custom underlining
%\usepackage{xunicode} % Allows generation of unicode characters from accented glyphs
\defaultfontfeatures{Mapping=tex-text} % Converts LaTeX specials (``quotes'' --- dashes etc.) to unicode

\usepackage{marginnote} % For margin years
\newcommand{\years}[1]{\marginnote{\scriptsize #1}} % New command for including margin years
\renewcommand*{\raggedleftmarginnote}{}
\setlength{\marginparsep}{7pt} % Slightly increase the distance of the margin years from the contant
\reversemarginpar

\usepackage{comment} % Comment out some lines
\usepackage{url}

%\usepackage[xetex, bookmarks, colorlinks, breaklinks, pdftitle={Takahiro Toriyabe - vita},pdfauthor={Takahiro Toriyabe }]{hyperref} % PDF setup - set your name and the title of the document to be incorporated into the final PDF file meta-information
%\hypersetup{linkcolor=blue,citecolor=blue,filecolor=black,urlcolor=MidnightBlue} % Link colors

%----------------------------------------------------------------------------------------
%	FONT CONFIGURATIONS
%----------------------------------------------------------------------------------------

% Two font choices are available in this template, the default is Linux Libertine, available for free at: http://www.linuxlibertine.org while the secondary choice is Hoefler Text which comes bundled with Mac OSX.
% To use Hoefler Text, comment out the Linux Libertine block below and uncomment the Hoefler Text block. You will also need to replace the "\&" characters with "\amper{}" in section titles.

% Linux Libertine Font (default)
%\setromanfont [Ligatures={Common}, Numbers={OldStyle}, Variant=01]{Linux Libertine O} % Main text font
%\setmonofont[Scale=0.8]{Monaco} % Set mono font (e.g. phone numbers)
%\sectionfont{\mdseries\upshape\Large} % Set font options for sections
%\subsectionfont{\mdseries\scshape\normalsize} % Set font options for subsections
%\subsubsectionfont{\mdseries\upshape\large} % Set font options for subsubsections
%\chardef\&="E050 % Custom ampersand character

% Hoefler Text Font (bundled with Mac OSX)
%\setromanfont [Ligatures={Common}, Numbers={OldStyle}]{Hoefler Text} % Main text font
%\setmonofont[Scale=0.8]{Monaco} % Set mono font (e.g. phone numbers)
%\setsansfont[Scale=0.9]{Optima Regular} % Set sans font, used in the main name and titles in the document
%\newcommand{\amper}{{\fontspec[Scale=.95]{Hoefler Text}\selectfont\itshape\&}} % Custom ampersand character
%\sectionfont{\sffamily\mdseries\large\underline} % Set font options for sections
%\subsectionfont{\rmfamily\mdseries\scshape\normalsize} % Set font options for subsections
%\subsubsectionfont{\rmfamily\bfseries\upshape\normalsize} % Set font options for subsubsections

% Original font set based on Linux Libertine Font
\setromanfont [Ligatures={Common}, Variant=01]{Linux Libertine O} % Main text font
%\setmonofont[Scale=0.8]{Monaco} % Set mono font (e.g. phone numbers)
\sectionfont{\bfseries\upshape\Large} % Set font options for sections
\subsectionfont{\bfseries\scshape\large} % Set font options for subsections
\subsubsectionfont{\bfseries\upshape\large} % Set font options for subsubsections

%----------------------------------------------------------------------------------------

\begin{document}

%----------------------------------------------------------------------------------------
%	CONTACT AND GENERAL INFORMATION SECTION
%----------------------------------------------------------------------------------------

{\LARGE Takahiro Toriyabe}\\[1cm] % Your name

\section*{Contact Information}
Graduate School of Economics, The University of Tokyo, Hongo 7-3-1, Bunkyo-ku, Tokyo \texttt{113-0033}, Japan\\
%Phone: \texttt{609-734-8000}\\ % Your phone number
%Fax: \texttt{609-924-8399}\\[.2cm] % Your fax number
Email: ttoriyabe@g.ecc.u-tokyo.ac.jp\\ % Your email address
\textsc{url}: \url{https://sites.google.com/site/takatoriyabe/}\\ % Your academic/personal website

%\vfill % Whitespace between contact information and specific CV information

%------------------------------------------------

\subsection*{Personal Information}

Date of Birth: July 15, 1993\\ % Your date of birth
Citizenship: Japan\\ % Your nationality
Language: Japanese (Native), English (Fluent)

%------------------------------------------------

\section*{Current Position}

\emph{Ph.D Student}, Graduate School of Economics, The University of Tokyo\\
\emph{Research Fellow}, The Japan Society for the Promotion of Science

%------------------------------------------------

\section*{Research fields}

Labor Economics, Family Economics, Applied Econometrics, National Transfer Accounts. % Your primary areas of research interest

%----------------------------------------------------------------------------------------
%	WORK EXPERIENCE SECTION
%----------------------------------------------------------------------------------------

%----------------------------------------------------------------------------------------
%	EDUCATION SECTION
%----------------------------------------------------------------------------------------

\section*{Education}

\years{2018}\textsc{MA} in Economics, Graduate School of Economics, The University of Tokyo\\
\years{2016}\textsc{BA} in Economics, Hitotsubashi University

%----------------------------------------------------------------------------------------
%	GRANTS, HONORS AND AWARDS SECTION
%----------------------------------------------------------------------------------------

\section*{Grants, Honors \& Awards}

\subsection*{Grants}

\years{2018--2020} Research Fellow of the Japan Society for the Promotion of Science (DC1)

\subsection*{Honor \& Awards}

\years{2017}Best Paper Award, 12th Applied Econometrics Conference, Hitotsubashi University\\
\years{2016}Best Poster Award, 19th Labor Economics Conference, Osaka University\\
\years{2015}Excellent Students Award, Hitotsubashi University\\
\years{2014}Excellent Students Award, Hitotsubashi University

%----------------------------------------------------------------------------------------
%	PUBLICATIONS AND TALKS SECTION
%----------------------------------------------------------------------------------------

\section*{Publications}

%\subsection*{Journal Articles}

%\years{1901}Einstein, Albert (1901), “Folgerungen aus den Capillaritätserscheinungen (Conclusions Drawn from the Phenomena of Capillarity)", \emph{Annalen der Physik} 4: 513

%------------------------------------------------

%\subsection*{Books}

%\years{1954}Einstein, Albert (1954), \emph{Ideas and Opinions}, New York: Random House, ISBN 0-517-00393-7

%-----------------------------------------------

\subsection*{Working paper}

\years{2018} ``Parental Leaves and Female Skill Utilization: Evidence from PIAAC,'' \textit{RIETI Discussion Paper}, 18-E-003, Co-author: Daiji Kawaguchi


%------------------------------------------------

\subsection*{Work in Progress}

\years{} ``Empowerment Effects and Intertemporal Commitment of Married Couples: Evidence from Japanese Pension Reform''\\
\years{} ``Does a generous parental leave policy suppress career advancement of skilled women?'' Co-author: Daiji Kawaguchi

%-------------------------------------------------

\section*{Presentations}

\subsection*{Invited Presentation}

\years{2018} Chuo University

\subsection*{Conference Presentations}

\years{2018} 21st Labor Economics Conference, Kyoto\\
\years{2018} Japanese Economic Association Spring Meeting, Kobe \\
\years{2017} The Asian and Australasian Society of Labor Economics Conference 2017, Canberra \\
\years{2017} 12th Applied Econometrics Conference, Tokyo\\
\years{2016} 19th Labor Economics Conference, Osaka\\


%----------------------------------------------------------------------------------------
%	TEACHING SECTION
%----------------------------------------------------------------------------------------

\section*{Teaching}

\years{2018--2019} Teaching assistant of Econometrics, Graduate School of Public Policy, The University of Tokyo, Japan\\
\years{2017--2018} Teaching assistant of Econometrics II, Graduate School of Economics, The University of Tokyo\\
\years{2017} Teaching assistant of Econometrics, Graduate School of Public Policy, The University of Tokyo\\
\years{2017} Math Camp, Graduate School of Economics, The University of Tokyo\\
\years{2015} Teaching assistant of Labor Economics, Hitotsubashi University

%------------------------------------------------

%\section*{Service to the profession}


\vfill{} % Whitespace before final footer

%----------------------------------------------------------------------------------------
%	FINAL FOOTER
%----------------------------------------------------------------------------------------

\begin{center}
{\scriptsize Last updated: \today} % Any final footer text such as a URL to the latest version of your CV, last updated date, compiled in XeTeX, etc
\end{center}

%----------------------------------------------------------------------------------------

\end{document}